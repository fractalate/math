\documentclass[12pt]{article}

\usepackage{amsmath}
\usepackage{amsfonts}
\usepackage{amssymb}

\title{Problems from 2024}
\author{Kate Novak}
\date{\today}

\begin{document}

\maketitle

\begin{abstract}
Welcome to my 2024 problem set, a collection of math problems that I encountered and found interesting throughout the year.
I hope you find them challenging and enjoy solving these as much as I did!
\end{abstract}

% Nothing for January, February, or March.
\setcounter{section}{3}
\section{April}

\subsection{Problem 1}

This problem has come from the 2023 GRE Mathematics Test Practice Book available from ETS.

In the complex plane, let $C$ be the circle $\{ z = 1 + e^{i\theta} : 0 \le \theta \le 2\pi\}$, oriented counterclockwise. What is the value of $\frac{1}{2 \pi i} \int_{C} \left( \frac{\sin z}{z - 1} \right)^2\,dz$?

\subsubsection*{Solution}

Let $f(z) = \left( \frac{\sin z}{z - 1} \right)^2$, which is a holomorphic function with a singularity at $z=1$. The contour integral over the circle can be calculated by leveraging the {\em{residue theorem}} which in our case for one singularity states

\begin{align}
  \int_{C} f(z)\, dz = 2 \pi i \, I(C,1) \, \operatorname{Res}(f,1) = 2 \pi i \, \operatorname{Res}(f,1)
\end{align}

$I(C,1)$ is the winding number at $z=1$, which is equal to 1 since the curve $C$ winds around the point counterclockwise a single time time.
$\operatorname{Res}(f,1)$ is the residue of $f$ at $z=1$ which is found by examining the coefficients of the laurent series expansion of $f(z)$ at $z = 1$:

\begin{align}
  f(z) =
  \frac{\sin^2(1)}{(z-1)^2} + \frac{\sin(2)}{z-1} + \cos(2) + \frac{2}{3}(z-1)\sin(2) + \dots
\end{align}

We find $\operatorname{Res}(f,1) = \sin(2)$, the coefficient of the $\frac{1}{z-1}$ term.
Then the integral from the problem can be evaluated

\begin{align}
  \frac{1}{2 \pi i} \int_{C} f(z)\,dz
    = \frac{2 \pi i}{2 \pi i} \operatorname{Res}(f,1)
    = \operatorname{Res}(f,1)
    = \sin(2)
    \approx 0.708.
\end{align}

\end{document}
