\documentclass[12pt]{article}

\usepackage{amsmath}
\usepackage{amsfonts}
\usepackage{amssymb}

\title{Problems from 2024 (Draft)}

\begin{document}

\maketitle

Please enjoy this collection of math problems that I encountered and found interesting throughout the year.

% Nothing for January, February, or March.
\setcounter{section}{3}
\section{April}

\subsection{Problem Statement}

This problem comes from the 2023 GRE Mathematics Test Practice Book available from ETS.

In the complex plane, let $C$ be the circle $\{ z = 1 + e^{i\theta} : 0 \le \theta \le 2\pi\}$, oriented counterclockwise. What is the value of $\frac{1}{2 \pi i} \int_{C} \left( \frac{\sin z}{z - 1} \right)^2\,dz$?

\subsubsection*{Solution}

Let $f(z) = \left( \frac{\sin z}{z - 1} \right)^2$ which is a holomorphic function with a {\em{pole}} (or {\em{non-essential singularity}}) at $z=1$ with order 2. The contour integral over the circle can be calculated by leveraging the {\em{residue theorem}} which in our case for one pole states

\begin{align}
  \int_{C} f(z)\, dz = 2 \pi i \, I(C,1) \, \operatorname{Res}(f,1) = 2 \pi i \, \operatorname{Res}(f,1).
\end{align}

$I(C,1)$ is the winding number at $z=1$, which is equal to 1 since the curve $C$ winds around the point counterclockwise a single time. $\operatorname{Res}(f,1)$ is the residue of $f$ at $z=1$ which can be found via the {\em{limit formula for higher-order poles}} with $n = 2$ to match the degree of the pole

\begin{align}
  \operatorname{Res}(f, z_0 = 1)
    &= \frac{1}{(n-1)!} \lim_{z \rightarrow z_0} \, \frac{d^{n-1}}{dz^{n-1}} \left[ \left(z - z_0\right)^n \cdot f(z) \right] \\
    &= \lim_{z \rightarrow 1} \, \frac{d}{dz} \left[ \left(z - 1\right)^2 \cdot \frac{\sin^2 z}{(z-1)^2} \right] \\
    &= \lim_{z \rightarrow 1} 2 \sin z \cos z \\
    &= 2 \sin 1 \cos 1 \\
    &= \sin 2.
\end{align}

Now the integral from the problem can be evaluated

\begin{align}
  \frac{1}{2 \pi i} \int_{C} f(z)\,dz
    = \frac{2 \pi i}{2 \pi i} \operatorname{Res}(f,1)
    = \operatorname{Res}(f,1)
    = \sin(2)
    \approx 0.708.
\end{align}

\subsection{Problem Statement}

This problem comes from chapter 3, exercise 45 in Contemporary Abstract Algebra, 7th Edition by Joseph A. Gallian.

Let $H$ be a subgroup of a finite group $G$. Suppose that $g$ belongs to $G$ and $n$ is the smallest positive integer such that $g^n \in H$. Prove that $n$ divides $|g|$.

\subsubsection*{Proof}

Let $m = |g|$. For the sake of contradiction, suppose $n \nmid m$ so that $m = kn + r$ for some $k \in \mathbb{Z}^{\ge 0}$ and $r \in \mathbb{Z}$ such that $0 < r < m$. Then, for any $j \in \mathbb{Z}$

\begin{align}
  m &= kn + r \\
    &= kn + r + jn - jn \\
    &= (k + j)n + r - jn
\end{align}

so

\begin{align}
  e = g^m &= g^{(k + j)n + r - jn} \\
          &= g^{(k + j)n}g^{r - jn}
\end{align}

so

\begin{align}
  g^{-(k + j)n} = g^{r - jn} \in H
\end{align}

Choose $j$ to be the greatest integer such that $0 \le r - jn < n$. If $r - jn = 0$ then $m = (k + j)n$, but this can't be the case since $n \nmid m$, so $0 < r - jn < n$. However, $n$ is the least positive integer such that $g^n \in H$ so we have a contradiction. Therefore $n|m$, so $n$ divides $|g|$.


\end{document}
