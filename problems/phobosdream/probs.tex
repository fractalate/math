\documentclass[12pt]{article}

\usepackage{amsmath}
\usepackage{amsfonts}
\usepackage{amssymb}
\usepackage{pgfplots}
\pgfplotsset{compat=1.16}

\title{Problems from 2024 (Draft)}

\begin{document}

\maketitle

Please enjoy this collection of math problems that I encountered and found interesting throughout the year.

% Nothing for January, February, or March.
\setcounter{section}{3}
\section{April}

\subsection{Problem Statement}

This problem comes from the 2023 GRE Mathematics Test Practice Book available from ETS.

In the complex plane, let $C$ be the circle $\{ z = 1 + e^{i\theta} : 0 \le \theta \le 2\pi\}$, oriented counterclockwise. What is the value of $\frac{1}{2 \pi i} \int_{C} \left( \frac{\sin z}{z - 1} \right)^2\,dz$?

\subsubsection*{Solution}

Let $f(z) = \left( \frac{\sin z}{z - 1} \right)^2$ which is a holomorphic function with a {\em{pole}} (or {\em{non-essential singularity}}) at $z=1$ with order 2. The contour integral over the circle can be calculated by leveraging the {\em{residue theorem}} which in our case for one pole states

\begin{align}
  \int_{C} f(z)\, dz = 2 \pi i \, I(C,1) \, \operatorname{Res}(f,1) = 2 \pi i \, \operatorname{Res}(f,1).
\end{align}

$I(C,1)$ is the winding number at $z=1$, which is equal to 1 since the curve $C$ winds around the point counterclockwise a single time. $\operatorname{Res}(f,1)$ is the residue of $f$ at $z=1$ which can be found via the {\em{limit formula for higher-order poles}} with $n = 2$ to match the degree of the pole

\begin{align}
  \operatorname{Res}(f, z_0 = 1)
    &= \frac{1}{(n-1)!} \lim_{z \rightarrow z_0} \, \frac{d^{n-1}}{dz^{n-1}} \left[ \left(z - z_0\right)^n \cdot f(z) \right] \\
    &= \lim_{z \rightarrow 1} \, \frac{d}{dz} \left[ \left(z - 1\right)^2 \cdot \frac{\sin^2 z}{(z-1)^2} \right] \\
    &= \lim_{z \rightarrow 1} 2 \sin z \cos z \\
    &= 2 \sin 1 \cos 1 \\
    &= \sin 2.
\end{align}

Now the integral from the problem can be evaluated

\begin{align}
  \frac{1}{2 \pi i} \int_{C} f(z)\,dz
    = \frac{2 \pi i}{2 \pi i} \operatorname{Res}(f,1)
    = \operatorname{Res}(f,1)
    = \sin(2)
    \approx 0.708.
\end{align}

\subsection{Problem Statement}

This problem comes from chapter 3, exercise 45 in Contemporary Abstract Algebra, 7th Edition by Joseph A. Gallian.

Let $H$ be a subgroup of a finite group $G$. Suppose that $g$ belongs to $G$ and $n$ is the smallest positive integer such that $g^n \in H$. Prove that $n$ divides $|g|$.

\subsubsection*{Proof}

Let $m = |g|$. For the sake of contradiction, suppose $n \nmid m$ so that $m = kn + r$ for some $k \in \mathbb{Z}^{\ge 0}$ and $r \in \mathbb{Z}$ such that $0 < r < m$. Then, for any $j \in \mathbb{Z}$

\begin{align}
  m &= kn + r \\
    &= kn + r + jn - jn \\
    &= (k + j)n + r - jn
\end{align}

so

\begin{align}
  e = g^m &= g^{(k + j)n + r - jn} \\
          &= g^{(k + j)n}g^{r - jn}
\end{align}

so

\begin{align}
  g^{-(k + j)n} = g^{r - jn} \in H
\end{align}

Choose $j$ to be the greatest integer such that $0 \le r - jn < n$. If $r - jn = 0$ then $m = (k + j)n$, but this can't be the case since $n \nmid m$, so $0 < r - jn < n$. However, $n$ is the least positive integer such that $g^n \in H$ so we have a contradiction. Therefore $n|m$, so $n$ divides $|g|$.

\section{May}

\subsection{Problem Statement}

Find the area enclosed by an astroid, a curve given by the following parametric equations:
\begin{align}
  x &= a \cos^3 \theta \\
  y &= a \sin^3 \theta 
\end{align}
for some constant $a$.

\subsubsection*{Solution}

The curve for an astroid looks like this:

\vspace{.5cm}

\begin{center}
\begin{tikzpicture}
  \begin{axis}[
      axis lines=center,
      axis equal,
      xmin=-1.5, xmax=1.5,
      ymin=-1.5, ymax=1.5,
      xlabel={$x$},
      ylabel={$y$},
      ticks=none
  ]
    % Draw the astroid curve using parametric plot
    \addplot [
        domain=0:2*pi,
        samples=100,
        thick,
        black,
    ] ({cos(deg(x))^3}, {sin(deg(x))^3});
    \node[label={below:$a$},circle,fill,inner sep=2pt] at (axis cs:1.0,0.0) {};
    \node[label={below:$-a$},circle,fill,inner sep=2pt] at (axis cs:-1.0,0.0) {};
  \end{axis}
\end{tikzpicture}
\end{center}
Notice that it is symmetric about the $x$-axis so we can find the area of the top half and then double it. The total area can be found by way of the following integral
\begin{align}
  A &= 2 \int_{-a}^{a} y \, dx = 2 \int_{\pi}^{0} y \frac{dx}{d\theta} \, d\theta
\end{align}
since $x = a$ when $\theta = 0$ and $x = -a$ when $\theta = \pi$ and where
\begin{align}
  \frac{dx}{d\theta} &= - 3a \cos^2 \theta \, \sin \theta.
\end{align}
Then
\begin{align}
  A &= 2 \int_\pi^0 a \sin^3 \theta \left( - 3a \cos^2 \theta \, \sin \theta \right) \, d\theta \\
    &= -6a^2 \int_\pi^0 \sin^4 \theta \, \cos^2 \theta \, d\theta \\
    &= 6a^2 \int_0^\pi \sin^2 \theta \, \left(\sin \theta \, \cos \theta\right)^2 \, d\theta \\
    &= 6a^2 \int_0^\pi \frac{1 - \cos 2\theta}{2} \, \left(\frac{1}{2} \sin 2\theta\right)^2 \, d\theta \\
    &= \frac{6a^2}{8} \int_0^\pi \left( 1 - \cos 2\theta \right) \, \sin^2 2\theta \, d\theta \\
    &= \frac{6a^2}{8} \int_0^\pi \sin^2 2\theta - \sin^2 2\theta \cos 2\theta \, d\theta \\
    &= \frac{6a^2}{8} \int_0^\pi \frac{1 - \cos 4\theta}{2} - \sin^2 2\theta \cos 2\theta \, d\theta \\
    &= \frac{6a^2}{8} \left[ \int_0^\pi \frac{1}{2} - \frac{1}{2}\cos 4\theta - \sin^2 2\theta \cos 2\theta \, d\theta \right] \\
    &= \frac{6a^2}{8} \left[\frac{1}{2} \theta - \frac{1}{8}\sin 4\theta - \frac{1}{6} \sin^3 2\theta \right]_0^\pi \\
    &= \frac{6a^2}{8} \left[\frac{1}{2} \pi - 0 \right] \\
    &= \frac {3 \pi a^2}{8}.
  \end{align}
\end{document}
